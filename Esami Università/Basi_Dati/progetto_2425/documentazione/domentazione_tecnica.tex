\documentclass[a4paper,12pt]{article}

\usepackage[italian]{babel}
\usepackage[T1]{fontenc}
\usepackage[utf8]{inputenc}
\usepackage{lmodern}
\usepackage{geometry}
\geometry{margin=2.5cm}
\usepackage{setspace}
\onehalfspacing
\usepackage{graphicx}
\usepackage{float}
\usepackage{listings}
\usepackage{xcolor}

\usepackage{fancyhdr}
\pagestyle{fancy}
\fancyhf{}
\lhead{ComicGalaxy - Documentazione Tecnica}
\rhead{41253A Federico Falcone} 
\cfoot{\thepage}
\lstset{
    language=PHP,
    basicstyle=\ttfamily\small,
    keywordstyle=\color{blue},
    stringstyle=\color{green!60!black},
    commentstyle=\color{gray},
    breaklines=true,
    frame=single
}

\begin{document}

\begin{titlepage}
    \centering
    \vspace*{2cm}

    {\LARGE \textbf{Documentazione Tecnica}}\\[0.5cm]
    {\Large Progetto Corso "Basi di dati"}\\[1.5cm]

    {\Huge \textbf{ComicGalaxy}}\\[2cm]

    \vfill

    \textbf{Autore:} Federico Falcone\\
    \textbf{Anno Accademico:} 2024--2025

\end{titlepage}

\tableofcontents
\newpage

\section{Introduzione}
ComicGalaxy è una catena di negozi specializzati nella vendita di fumetti e merchandise correlato. Attraverso questa applicazione web, è possibile, in qualità di cliente,
effettuare acquisti e iscriversi al programma federlta. È presente anche una sezione per i manager che consente loro di gestire il negozio di cui è responsabile, le utenze, le tessere fedeltà, i fornitore e visualizzare dei report.
\section{Schema concettuale}
\begin{figure}[H]
    \centering
    \includegraphics[width=1.10\textwidth]{er.png}
\end{figure}


L'utente è una generalizzazione totale ed esclusiva di cliente e manager, pertanto ogni utente può essere o un cliente o un manager, non entrambi.

Un negozio è gestito da una manager e un manager può gestire solo un negozio alla volta. Egli esegue tutte le attività di amministrazione del negozio, come modificare l'orario del negozio, effettuare ordini e visualizzarne i dati.   

Un negozio mette a disposizione alla vendita una serie di prodotti, definiti dalla relazione fornitura\_negozio: se il prezzo non è stabilito (\texttt{null}) il prodotto non è disponibile all'acquisto, in caso contrario sì. Inoltre, un negozio
ha un orario di apertura e chiusura e un indirizzo. Per rifornirsi, un negozio effettua un ordine presso un fornitore, il quale ha una disponibilità limitata di prodotto. L'ordine viene effettuato in base ad un criterio di economicità, ovvero l'ordine
viene effettuato presso il fornitore che offre il prezzo più basso per il prodotto richiesto. In caso la richiesta di un prodotto sia superiore a quella disponibile presso un fornitore, la rimanente quantità di prodotto viene ordinata presso il fornitore che offre il prezzo più basso tra quelli rimanenti.
L'ordine poi, dopo 3 giorni, verrà consegnato e il negozio dovrà ritirarlo.

Un cliente effettua acquisti presso un determinato negozio. Possono inoltre richiedere una tessera fedeltà presso quel negozio, utilizzabile per tutti i punti vendita della catena, la quale permette di accumulare punti ad ogni acquisto, tramite i quali
potrà accedere a sconti dedicati. Un cliente può possedere al massimo una tessera. In caso la tessera scada egli potrà rinnovarla presso qualsiasi negozio, ma in caso essa venga sospesa, non potrà nè essere rinnovata nè utilizzata.

I manager, oltre a gestire il proprio punto vendita, può gestire una serie di operazioni "comuni" dell'applicazione, come creare nuove utenze per clienti o manager, riassegnare manager ad altri negozi, visualizzare i clienti tesserati, gestire i fornitori e visualizzare i report.

Gli utenti accedono al sistema tramite autenticazione con e-mail e password. Grazie alla generalizzazione totale ed esclusiva, il sistema riconosce se l'utente che ha effettuato l'accesso sia un manager o un cliente.

\section{Schema logico}
\begin{itemize}
    \item \textbf{Utente}(\underline{mail}, telefono, password,sospeso). \\ Sospeso è di tipo booleano e inizialmente vale \texttt{false};
    \item \textbf{Manager}(\underline{mail}[FK $\rightarrow$ Utente ], nome, cognome);
    \item \textbf{Cliente}(\underline{cf}, mail[FK $\rightarrow$ Utente ], nome, cognome);
    \item \textbf{Indirizzo}{\underline{id}, citta, via, civico};
    \item \textbf{Negozio}(\underline{id}, nome, telefono\textsuperscript{*}, manager[FK $\rightarrow$ Manager], id\_indirizzo [FK $\rightarrow$ Indirizzo], data\_chiusura\textsuperscript{*});
    \item \textbf{Orario}(\underline{giorno}, \underline{id\_negozio} [FK $\rightarrow$ Negozio], ora\_apertura, ora\_chiusura);
    \item \textbf{Prodotto}{\underline{id},nome,descrizione};
    \item \textbf{Fornitura\_negozio}(\underline{id\_prodotto} [FK $\rightarrow$ Prodotto], \underline{id\_negozio} [FK $\rightarrow$ Negozio], prezzo, quantita);
    \item \textbf{Fornitore}(\underline{p\_iva},telefono, mail, indirizzo [FK $\rightarrow$ Indirizzo] nome, sospeso). \\ Sospeso è di tipo booleano e inizialmente vale \texttt{false};
    \item \textbf{Fornitura\_fornitore}(\underline{p\_iva\_fornitore} [FK $\rightarrow$ Fornitore], \underline{id\_prodotto} [FK $\rightarrow$ Prodotto], prezzo, quantita);
    \item \textbf{Ordine}(\underline{id}, data\_consegna, negozio [FK $\rightarrow$ Negozio], fornitore [FK $\rightarrow$ Fornitore], ritirato, data). \\ Ritirato è di tipo booleano e inizialmente vale \texttt{false}, indica che l'ordine, una volta consegnato, è stato ritirato (e quindi aggiunto alla fornitura del negozio);
    \item \textbf{Dettaglio\_ordine}(\underline{id\_ordine} [FK $\rightarrow$ Ordine], \underline{id\_prodotto} [FK $\rightarrow$ Prodotto], quantita, prezzo\_unitario);
    \item \textbf{Tessera}(\underline{id}, saldo, cf\_cliente [FK $\rightarrow$ Cliente ], data\_emissione, data\_scadenza,  sospeso). \\ Sospeso è di tipo booleano e inizialmente vale \texttt{false};
    \item \textbf{Fattura}(\underline{id}, data\_acquisto, sconto, totale, cf\_cliente [FK $\rightarrow$ Cliente ], codice\_negozio [FK $\rightarrow$ Negozio ]);
    \item \textbf{Dettaglio\_fattura}(\underline{id\_fattura} [FK $\rightarrow$ Fattura], \underline{id\_prodotto} [FK $\rightarrow$ Prodotto], quantita, prezzo);
\end{itemize}
\section{Funzioni, trigger, viste}
Le funzioni, trigger e viste implementate sono fondamentali per il corretto funzionamento dell'applicazione web. Le funzioni permettono di eseguire operazioni atomiche, come l'acquisto di una serie di prodotti o creare le utenze per manager e clienti.
I trigger permettono di mantenere l'integrità dei dati, ad esempio impedendo che ad un cliente venga associata una mail già in uso da un altro utente. Altra funzione utile dei trigger è la possibilità di automatizzare degli aggiornamenti al verificarsi di un evento. Le viste invece facilitano l'accesso a dati più complessi, come ad esempio la lista dei clienti con la relativa tessera.
\\ In tutte le foreign key la regola di delete è settata su \texttt{RESTRICT} al fine di mantenere uno storico di tutte le operazioni effettuate. La cancellazione logica è comunque implementata tramite l'attributo \texttt{sospeso}.
\subsection{Funzioni}
\begin{itemize}
    \item \textbf{aggiorna\_fornitore(old\_iva, new\_iva, nome, telefono, mail, via, civico, città)} : modifica le informazioni di un fornitore;
    \item \textbf{aggiorna\_negozio(id, nome, città, via, civico, telefono)}: modifica le informazioni di un negozio;
    \item \textbf{aggiorna\_utente(cf, old\_mail, mail, nome, cognome, telefono)}: modifica le informazioni di un utente. Se l'utente è un cliente, aggiorna anche il CF;
    \item \textbf{calcola\_sconto(sconto, totale, cf\_cliente)}: restituisce lo sconto effettuato sul totale;
    \item \textbf{crea\_cliente(mail, nome, cognome, telefono, password, cf)}: crea un nuovo utente cliente;
    \item \textbf{crea\_fornitore(nome, p\_iva, mail, città, via, civico, telefono)}: crea un nuovo fornitore;
    \item \textbf{crea\_manager(mail, nome, cognome, telefono, password)}: crea un nuovo utente manager;
    \item \textbf{crea\_negozio(nome, manager, città, via, civico, telefono)}: crea un nuovo negozio;
    \item \textbf{crea\_tessera(mail, negozio)}: crea una nuova tessera per il cliente con la mail specificata;
    \item \textbf{gestisci\_acquisto(cf\_cliente, id\_negozio, prodotti[], quantità[], sconto)}: esegue l'acquisto dei prodotti specificati, applicando lo sconto scelto;
    \item \textbf{ordina\_prodotti(id\_negozio, prodotti[], quantità[])}: esegue l'ordine dei prodotti specificati scegliendo il fornitore più economico per ogni prodotto. Verrà poi eseguito un singolo ordine per fornitore;
    \item \textbf{riepilogo\_prodotto(id\_prodotto,quantità)}: restituisce il subtotale per un prodotto di una certa quantità in un ordine presso un fornitore;
    \item \textbf{rinnova\_tessera(cf)}: rinnova la tessera di un cliente;
\end{itemize}
\subsection{Trigger}
\begin{itemize}
    \item \textbf{trg\_aggiorna\_disponibilita\_fornitore}: dopo la creazione di un ordine e quindi dopo l'inserimento dei valori nella tabella \texttt{dettaglio\_ordini}, aggiorna la quanità di prodotto disponibile presso un fornitore;
    \item \textbf{trg\_aggiorna\_saldo\_punti}: dopo un acquisto da parte di un cliente, e quindi l'inserimento dei valori nella tabella \texttt{fattura}, aggiorna il saldo punti del cliente;
    \item \textbf{trg\_ritiro}: quando si ritira un ordine, ovvero quando si setta a \texttt{true} il valore \texttt{ritirato} presente nella tabella \texttt{ordine}, i prodotti presenti nell'ordine vengono aggiunti alla fornitura del negozio;
    \item \textbf{trg\_check\_manager\_disponibile}: evita che un manager associato ad un altro negozio venga associato ad un altro;
    \item \textbf{trg\_check\_orario}: impedisce di inserire un orario non valido;
    \item \textbf{trg\_check\_modifica\_data}: impedisce di modificare la data di consegna di un ordine con un valore non valido oppure se l'ordine è da ritirare;
    \item \textbf{trg\_chiusura\_negozio}: impedisce di aprire un negozio chiuso;
    \item \textbf{trg\_check\_cf}: impedisce di inserire o modificare un cf con un formato non valido;
    \item \textbf{trg\_check\_email}: impedisce di inserire o modificare una mail in un formato non valido;
    \item \textbf{trg\_check\_telefono}: impedisce di inserire o modificare un telefono in un formato non valido;
    \item \textbf{trg\_check\_p\_iva}: impedisce di inserire o modificare una p\_iva in un formato non valido;
    \item \textbf{trg\_sospendi\_tessera}: quando un utente viene sospeso, viene sospesa anche la tessera. Viceversa, quando un utente viene riattivato, la tessera viene riattivata;
    \item \textbf{trg\_manager\_email}: impedisce di inserire o modificare la mail di un manager con una mail presente nella tabella cliente;
    \item \textbf{trg\_cliente\_email}: impedisce di inserire o modificare la mail di un cliente con una mail presente nella tabella manager.
\end{itemize}
\subsection{Viste}
\begin{itemize}
    \item \textbf{v\_clienti}: elenco dei clienti con le relative informazioni e la tessera associata (se presente);
    \item \textbf{v\_clienti\_punti\_elevati}: elenco dei clienti, con le relative informazioni, che hanno più di 300 punti;
    \item \textbf{v\_storico\_acquisti}: elenco delle fatture con relative informazioni di cliente e negozio;
    \item \textbf{v\_prodotti\_ordinabili}: lista dei prodotti ordinabili presso tutti i fornitori. Il prezzo viene indicato come la media dei prezzi dei vari fornitori;
    \item \textbf{v\_storico\_ordini\_fornitori}: elenco degli ordini effettuati dai negozi ai fornitori con le relative informazioni;
    \item \textbf{v\_storico\_prodotti\_ordine}: elenco dei prodotti contenuti negli ordini effettuati presso i fornitori con le relative informazioni;
    \item \textbf{v\_storico\_tessere}: elenco delle tessere emesse dai negozi cje sono stati chiusi;
    \item \textbf{v\_tessere}: elenco dei clienti tesserati contentente le relative informazioni e quelle del negozio che l'ha emessa;
\end{itemize}
\section{Prove di funzionamento}
\subsection{Pagina iniziale}
\begin{figure}[H]
    \centering
    \includegraphics[width=1\textwidth]{Images/pagina_inziale.png}
\end{figure}
Dalla pagina inziale un utente può visualizzare varie informazioni sulla catena tramite la toolbar in altro. Può visualizzare i negozi:
\begin{figure}[H]
    \centering
    \includegraphics[width=1\textwidth]{Images/negozi.png}
\end{figure}
Da cui può visualizzare l'orario di ogni negozio:
\begin{figure}[H]
    \centering
    \includegraphics[width=1\textwidth]{Images/orari.png}
\end{figure}
Oppure i prodotti disponibili e in quale negozio sono acquistabili.
\begin{figure}[H]

    \includegraphics[width=1\textwidth]{Images/prodotti.png}
    \includegraphics[width=1\textwidth]{Images/disponibilita.png}
\end{figure}
\pagebreak
\subsection{Login}
L'utente può effettuare il login tramite il pulsante in alto a destra:
\begin{figure}[H]
    \centering
    \includegraphics[width=0.5\textwidth]{Images/login.png}
\end{figure}
Ad avvenuto accesso, l'utente verrà reindirizzato alla propria area personale, che varierà a seconda del tipo di utente (cliente o manager).
\subsection{Area riservata}
\subsubsection{Area Manager}
In caso il manager sia stato sospeso si visualizzerà la seguente schermata:
\begin{figure}[H]
    \centering
    \includegraphics[width=1\textwidth]{Images/manager_sospeso.png}
\end{figure}
\pagebreak
All'accesso il manager visualizzerà la propria area riservata:
\begin{figure}[H]
    \centering
    \includegraphics[width=1\textwidth]{Images/area_manager.png}
\end{figure}
Da questa schermata, il manager può entrare nella gestione del proprio negozio oppure eseguire delle operazioni più "generali".
\\ \textbf{\textcolor{blue}{Gestione del negozio}}:
\begin{figure}[H]
    \centering
    \includegraphics[width=1\textwidth]{Images/gestione_negozio.png}
\end{figure}
Da qui il manager può eseguire le operazioni di gestione del proprio negozio. 
\\\textbf{Magazzino}: il manager può visualizzare i prodotti i magazzino:
\begin{figure}[H]
    \centering
    \includegraphics[width=1\textwidth]{Images/magazzino.png}
\end{figure}

Da qui può cambiare il prezzo di un prodotto, rimuoverlo dalla vendita o rimuoverlo completamente dal magazzino:
\begin{figure}[H]
    \centering
    \includegraphics[width=1\textwidth]{Images/modifica_prodotti.png}
\end{figure}
\textbf{Ordini}: il manager può visualizzare lo storico degli ordini effettuati e ritirare quelli consegnati:
\begin{figure}[H]
    \centering
    \includegraphics[width=1\textwidth]{Images/storico_ordini.png}
\end{figure}
Visualizzarne i dettagli:
\begin{figure}[H]
    \centering
    \includegraphics[width=1\textwidth]{Images/dettagli_ordine.png}
\end{figure}
Oppure può crearne uno nuovo:
\begin{figure}[H]
    \centering
    \includegraphics[width=1\textwidth]{Images/crea_ordine.png}
        \includegraphics[width=1\textwidth]{Images/aggiungi_carrello.png}

\end{figure}
Il manager aggiunge i prodotti che desidera ordinare nel carrello selezionando le quantità dei prodotti che vuole e aggiungendoli tramite il tasto apposito. Il resoconto sarà poi visualizzabile all'interno di esso:
\begin{figure}[H]
    \centering
    \includegraphics[width=1\textwidth]{Images/carrello.png}
\end{figure}
Una volta confermato l'ordine, esso verrà creato e consegnato dopo 3 giorni. In caso i prodotti siano disponibili da fornitori diversi, verrà creato un ordine per fornitore. Al terzo giorno l'ordine sarà da ritirare:
\begin{figure}[H]
    \centering
    \includegraphics[width=1\textwidth]{Images/ritiro.png}
\end{figure}
\begin{figure}[H]
    \centering
    \includegraphics[width=1\textwidth]{Images/ritirato.png}
\end{figure}
\begin{figure}[H]
    \centering
    \includegraphics[width=1\textwidth]{Images/stato_ritirato.png}
\end{figure}
\pagebreak
\textbf{Clienti tesserati}: da qui il manager può visualizzare i clienti la cui tessera è stata emessa dal proprio negozio. È possibile sospendergliela.
\begin{figure}[H]
    \centering
    \includegraphics[width=1\textwidth]{Images/tesserati.png}
        \includegraphics[width=1\textwidth]{Images/sospesa.png}

\end{figure}
\textbf{Fatture emesse}: è possibile visualizzare tutte le vendite effettuate dal negozio, con i relativi dettagli:
\begin{figure}[H]
    \centering
    \includegraphics[width=1\textwidth]{Images/fatture.png}
\end{figure}
\begin{figure}[H]
    \centering
    \includegraphics[width=1\textwidth]{Images/dettaglio_fattura.png}
\end{figure}
\pagebreak
\textbf{Chiusura negozio}: il manager può chiudere definitivamente il negozio tramite l'apposito tasto:
\begin{figure}[H]
    \centering
    \includegraphics[width=0.75\textwidth]{Images/chiusura.png}
    \includegraphics[width=1\textwidth]{Images/chiuso.png}

\end{figure}
Tutte le funzioni che permettevano di eseguire le varie operazioni sul negozio non sono più accessibili.\\
\pagebreak
\textbf{\textcolor{blue}{Operazioni generali}}
\\
\textbf{Gestione clienti}: da questa schermata il manager può visualizzare tutti i clienti, modificare i loro dati e sospenderli/riattivarli.
\begin{figure}[H]
    \centering
    \includegraphics[width=1\textwidth]{Images/gestione_clienti.png}
\end{figure}
\begin{figure}[H]
    \includegraphics[width=0.5\textwidth]{Images/modifica_cliente.png}
    \includegraphics[width=0.5\textwidth]{Images/crea_cliente.png}
    \end{figure}
Visualizzazione degli errori:
\begin{figure}[H]
    \centering
    \includegraphics[width=1\textwidth]{Images/errore_modifica.png}
\end{figure}
Quando si sospende un cliente, viene sospesa anche la tessera e viceversa:
\begin{figure}[H]
    \centering
    \includegraphics[width=1\textwidth]{Images/sospensione_cliente.png}
    \includegraphics[width=1\textwidth]{Images/riattiva_cliente.png}
\end{figure}
\pagebreak
\textbf{Gestione manager}:
Molto simile alla gestione clienti:
\begin{figure}[H]
    \centering
    \includegraphics[width=1\textwidth]{Images/gestione_manager.png}
\end{figure}
\begin{figure}[H]
    \includegraphics[width=0.5\textwidth]{Images/modifica_manager.png}
    \includegraphics[width=0.5\textwidth]{Images/crea_manager.png}
\end{figure}
\pagebreak
\textbf{Gestione negozi}: da qui un manger può assegnare un manager libero ad un negozio, sostituendo quello attuale, modificare le informazioni di un negozio oppure crearne uno nuovo.
\begin{figure}[H]
    \centering
    \includegraphics[width=1\textwidth]{Images/gestione_negozi.png}
\end{figure}
Riassegnazione di un manager:
\begin{figure}[H]
    \centering
    \includegraphics[width=1\textwidth]{Images/riassegnazione_manager1.png}
    \includegraphics[width=1\textwidth]{Images/riassegnazione_manager2.png}

\end{figure}
Creazione di un nuovo negozio:
\begin{figure}[H]
    \centering
    \includegraphics[width=1\textwidth]{Images/creazione_negozio.png}
    \includegraphics[width=1\textwidth]{Images/negozio_creato.png}    
\end{figure}
In caso si inserisse un indirizzo già occupato da un altro negozio: 
\begin{figure}[H]
    \centering
    \includegraphics[width=1\textwidth]{Images/errore_creazione_negozio.png}
\end{figure}
\pagebreak
Modifica di un negozio:
\begin{figure}[H]
    \centering
    \includegraphics[width=0.5\textwidth]{Images/modifica_negozio.png}
\end{figure}
\pagebreak
\textbf{Gestione fornitori}: da qui il manager può visualizzare tutti i fornitori, modificarne le informazioni, visualizzare i prodotti forniti da ciascuno e crearne uno nuovo.
\begin{figure}[H]
    \centering
    \includegraphics[width=1\textwidth]{Images/gestione_fornitori.png}
    \includegraphics[width=1\textwidth]{Images/prodotti_fornitore.png}
\end{figure}
Creazione di un nuovo fornitore:
\begin{figure}[H]
    \centering
    \includegraphics[width=1\textwidth]{Images/crea_fornitore.png}
    \includegraphics[width=1\textwidth]{Images/fornitore_creato.png}

\end{figure}
In caso di partita IVA già esistente:
\begin{figure}[H]
    \centering
    \includegraphics[width=1\textwidth]{Images/errore_fornitore.png}
\end{figure}
Modifica di un fornitore:
\begin{figure}[H]
    \centering
    \includegraphics[width=0.5\textwidth]{Images/modifica_fornitore.png}
\end{figure}
Sospensione:
\begin{figure}[H]
    \centering
    \includegraphics[width=1\textwidth]{Images/sospensione1.png}
    \includegraphics[width=1\textwidth]{Images/sospensione2.png}
\end{figure}
\pagebreak
\textbf{\textcolor{blue}{Report}}: nella sezione report il manager può visualizzare varie informazioni:
\begin{figure}[H]
    \centering
    \includegraphics[width=1\textwidth]{Images/report.png}
\end{figure}
\textbf{Tesserati per negozio}: da questa schermata il manager può visualizzare tutti i tesserati la cui tessera è stata emessa da un determinato negozio:
\begin{figure}[H]
    \centering
    \includegraphics[width=1\textwidth]{Images/tesserati_negozio.png}
    \includegraphics[width=1\textwidth]{Images/tesserati_negozio2.png}

\end{figure}
\pagebreak
\textbf{Tesserati punti elevati}: visualizza quali tesserati hanno un saldo punti maggiore di 300:
\begin{figure}[H]
    \centering
    \includegraphics[width=1\textwidth]{Images/tesserati_punti_elevati.png}
\end{figure}
\textbf{Ordini per fornitore}: visualizza tutti gli ordini effettuati presso un determinato fornitore:
\begin{figure}[H]
    \centering
    \includegraphics[width=1\textwidth]{Images/ordini_fornitore.png}
    \includegraphics[width=1\textwidth]{Images/ordini_fornitore2.png}
\end{figure}
\pagebreak
\subsubsection{Area Clienti}
Nel caso in cui l'account del cliente venga sospeso esso non potrà svolgere nessuna attività e verrà mostrata la seguente schermata:
\begin{figure}[H]
    \centering
    \includegraphics[width=1\textwidth]{Images/cliente_sospeso.png}
\end{figure}
Il cliente, accedendo alla propria area riservata, visualizzerà la seguente interfaccia:
\begin{figure}[H]
    \centering
    \includegraphics[width=1\textwidth]{Images/area_clienti.png}
\end{figure}
\textbf{Acquista}: da qui il cliente può effettuare acquisti presso un determinato negozio:
\begin{figure}[H]
    \centering
    \includegraphics[width=1\textwidth]{Images/acquisto_prodotti1.png}
    \includegraphics[width=1\textwidth]{Images/acquisto_prodotti2.png}

\end{figure}
Il cliente selezionerà i prodotti che intende acquistare con la loro quantità e li aggiungerà al carrello tramite il tasto presente in fondo alla pagina:
\begin{figure}[H]
    \centering
    \includegraphics[width=1\textwidth]{Images/acquisto_prodotto3.png}
    \includegraphics[width=1\textwidth]{Images/aggiunti_successo.png}

\end{figure}
Quando il cliente aprirà il carrello, visualizzerà la seguente schermata:
\begin{figure}[H]
    \centering
    \includegraphics[width=1\textwidth]{Images/carrello1.png}
\end{figure}
In caso il cliente non possieda una tessera, egli potrà richiederla tramite l'apposito pulsante. 
\\ Quando poi il cliente avrà una tessera, egli potrà scegliere, se ha abbastanza punti, quale sconto applicare.
\begin{figure}[H]
    \centering
    \includegraphics[width=1\textwidth]{Images/nosconto.png}
    \includegraphics[width=1\textwidth]{Images/sconto_applicato.png}

\end{figure}
Applicando lo sconto, il totale verrà aggiornato di conseguenza.
\begin{figure}[H]
    \centering
    \includegraphics[width=1\textwidth]{Images/totale.png}
\end{figure}
In caso di tessera scaduta o sospesa, non sarà possibile applicare gli sconti e guadagnare punti.
\begin{figure}[H]
    \centering
    \includegraphics[width=1\textwidth]{Images/scaduta.png}
    \includegraphics[width=1\textwidth]{Images/sospesa_tessera.png}

\end{figure}
Cliccando sul tasto \texttt{Conferma acquisto}, verrà eseguito l'acquisto, generando la fattura e aggiornando il saldo punti del cliente (se possiede una tessera valida).
\begin{figure}[H]
    \centering
    \includegraphics[width=1\textwidth]{Images/ordine_completato.png}
\end{figure}
\pagebreak
\textbf{I miei acquisti}: da qui il cliente può visualizzare lo storico di tutti gli acquisti effetuati.
\begin{figure}[H]
    \centering
    \includegraphics[width=1\textwidth]{Images/acquisti_cliente.png}
\end{figure}
Inoltre, cliccando sul codice della fattura, potrà visualizzarne i dettagli:
\begin{figure}[H]
    \centering
    \includegraphics[width=1\textwidth]{Images/dettaglio_acquisto.png}
\end{figure}
\textbf{Tessera}: da qui il cliente può visualizzare le informazioni relative alla propria tessera e rinnovarla in caso sia scaduta.
\begin{figure}[H]
    \centering
    \includegraphics[width=1\textwidth]{Images/tessera1.png}
\end{figure}
In caso sia scaduta:
\begin{figure}[H]
    \centering
    \includegraphics[width=1\textwidth]{Images/tessera2.png}
    \includegraphics[width=1\textwidth]{Images/tessera3.png}
\end{figure}

\end{document}
