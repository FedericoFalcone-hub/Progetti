\documentclass[a4paper,12pt]{article}

\usepackage[italian]{babel}
\usepackage[T1]{fontenc}
\usepackage[utf8]{inputenc}
\usepackage{lmodern}

\usepackage{geometry}
\geometry{margin=2.5cm}

\usepackage{setspace}
\onehalfspacing

\usepackage{graphicx}
\usepackage{float}

\usepackage{hyperref}
\hypersetup{
    colorlinks=true,
    linkcolor=black,
    urlcolor=blue
}

\usepackage{fancyhdr}
\pagestyle{fancy}
\fancyhf{}
\lhead{ComicGalaxy - Manuale Utente}
\rhead{41253A Federico Falcone} 
\cfoot{\thepage}

\begin{document}

\begin{titlepage}
    \centering
    \vspace*{2cm}

    {\LARGE \textbf{Manuale Utente}}\\[0.5cm]
    {\Large Progetto Corso "Basi di dati"}\\[1.5cm]

    {\Huge \textbf{ComicGalaxy}}\\[2cm] % ← nome progetto

    \vfill

    \textbf{Autore:} 41253A Federico Falcone\\
    \textbf{Anno Accademico:} 2024--2025\\
    \textbf{Tecnologie:} PHP, PostgreSQL, HTML, CSS, Bootstrap

    \vspace*{1cm}
\end{titlepage}

\tableofcontents
\newpage

\section{Introduzione}
ComicGalaxy è un'applicazione web che consente di gestire la propia catena di negozi. Al suo interno un cliente può effettuare acquisti ed iscriversi al programma fedeltà.
\\ Un manager invece può occuparsi di tutta la parte gestionale del negozio di cui è responsabile, gestire le utenze, tessere, negozi, fornitori e visualizzare le statistiche di tutto il sistema.
\\
Questo manuale fornisce le informazioni e istruzioni fondamentali per l'installazione dell'applicazione web ComicGalaxy.

\section{Requisiti di sistema}
Per eseguire ComicGalaxy, è necessario disporre dei seguenti requisiti di sistema:
\begin{itemize}
    \item Un server web compatibile con PHP (ad esempio Apache o Nginx).
    \item PHP versione 8.0 o superiore.
    \item Un database PostgreSQL versione 16 o superiore.
\end{itemize}
Estensioni PHP richieste:
\begin{itemize}
    \item \texttt{pdo\_pgsql} - Per la connessione al database PostgreSQL
    \item \texttt{session} - Per la gestione delle sessioni utente
    \item \texttt{mbstring} - Per la gestione delle stringhe multibyte
\end{itemize}
Per installare il necessario:\\
 \texttt{sudo apt update\\
sudo apt install apache2 php php-pgsql postgresql postgresql-contrib
\\sudo systemctl start apache2
\\sudo systemctl start postgresql
\\sudo systemctl enable apache2
\\sudo systemctl enable postgresql}
\\      \\
Per creare il database:\\
\texttt{sudo -u postgres psql}\\
\texttt{CREATE DATABASE ComicGalaxy;}\\
\texttt{CREATE USER ComicGalaxy\_user WITH PASSWORD 'password';}\\
\texttt{GRANT ALL PRIVILEGES ON DATABASE ComicGalaxy TO ComicGalaxy\_user;}
\section{Credenziali}
Per accedere all'applicazione web come manager sono state predisposte le seguenti credenziali:
\begin{itemize}
    \item Mail: \texttt{federico.falcone@comicgalaxy.it}
    \item Password: \texttt{FedericoFalcone}
\end{itemize}
Per accedere come cliente, invece:
\begin{itemize}
    \item Mail: \texttt{alice.rossi@example.it}
    \item Password: \texttt{AliceRossi}
\end{itemize}
\end{document}
